\PassOptionsToPackage{unicode=true}{hyperref} % options for packages loaded elsewhere
\PassOptionsToPackage{hyphens}{url}
%
\documentclass[]{article}
\usepackage{lmodern}
\usepackage{amssymb,amsmath}
\usepackage{ifxetex,ifluatex}
\usepackage{fixltx2e} % provides \textsubscript
\ifnum 0\ifxetex 1\fi\ifluatex 1\fi=0 % if pdftex
  \usepackage[T1]{fontenc}
  \usepackage[utf8]{inputenc}
  \usepackage{textcomp} % provides euro and other symbols
\else % if luatex or xelatex
  \usepackage{unicode-math}
  \defaultfontfeatures{Ligatures=TeX,Scale=MatchLowercase}
\fi
% use upquote if available, for straight quotes in verbatim environments
\IfFileExists{upquote.sty}{\usepackage{upquote}}{}
% use microtype if available
\IfFileExists{microtype.sty}{%
\usepackage[]{microtype}
\UseMicrotypeSet[protrusion]{basicmath} % disable protrusion for tt fonts
}{}
\IfFileExists{parskip.sty}{%
\usepackage{parskip}
}{% else
\setlength{\parindent}{0pt}
\setlength{\parskip}{6pt plus 2pt minus 1pt}
}
\usepackage{hyperref}
\hypersetup{
            pdftitle={CLV Exploratory},
            pdfauthor={Josh Kong},
            pdfborder={0 0 0},
            breaklinks=true}
\urlstyle{same}  % don't use monospace font for urls
\usepackage[margin=1in]{geometry}
\usepackage{color}
\usepackage{fancyvrb}
\newcommand{\VerbBar}{|}
\newcommand{\VERB}{\Verb[commandchars=\\\{\}]}
\DefineVerbatimEnvironment{Highlighting}{Verbatim}{commandchars=\\\{\}}
% Add ',fontsize=\small' for more characters per line
\usepackage{framed}
\definecolor{shadecolor}{RGB}{248,248,248}
\newenvironment{Shaded}{\begin{snugshade}}{\end{snugshade}}
\newcommand{\AlertTok}[1]{\textcolor[rgb]{0.94,0.16,0.16}{#1}}
\newcommand{\AnnotationTok}[1]{\textcolor[rgb]{0.56,0.35,0.01}{\textbf{\textit{#1}}}}
\newcommand{\AttributeTok}[1]{\textcolor[rgb]{0.77,0.63,0.00}{#1}}
\newcommand{\BaseNTok}[1]{\textcolor[rgb]{0.00,0.00,0.81}{#1}}
\newcommand{\BuiltInTok}[1]{#1}
\newcommand{\CharTok}[1]{\textcolor[rgb]{0.31,0.60,0.02}{#1}}
\newcommand{\CommentTok}[1]{\textcolor[rgb]{0.56,0.35,0.01}{\textit{#1}}}
\newcommand{\CommentVarTok}[1]{\textcolor[rgb]{0.56,0.35,0.01}{\textbf{\textit{#1}}}}
\newcommand{\ConstantTok}[1]{\textcolor[rgb]{0.00,0.00,0.00}{#1}}
\newcommand{\ControlFlowTok}[1]{\textcolor[rgb]{0.13,0.29,0.53}{\textbf{#1}}}
\newcommand{\DataTypeTok}[1]{\textcolor[rgb]{0.13,0.29,0.53}{#1}}
\newcommand{\DecValTok}[1]{\textcolor[rgb]{0.00,0.00,0.81}{#1}}
\newcommand{\DocumentationTok}[1]{\textcolor[rgb]{0.56,0.35,0.01}{\textbf{\textit{#1}}}}
\newcommand{\ErrorTok}[1]{\textcolor[rgb]{0.64,0.00,0.00}{\textbf{#1}}}
\newcommand{\ExtensionTok}[1]{#1}
\newcommand{\FloatTok}[1]{\textcolor[rgb]{0.00,0.00,0.81}{#1}}
\newcommand{\FunctionTok}[1]{\textcolor[rgb]{0.00,0.00,0.00}{#1}}
\newcommand{\ImportTok}[1]{#1}
\newcommand{\InformationTok}[1]{\textcolor[rgb]{0.56,0.35,0.01}{\textbf{\textit{#1}}}}
\newcommand{\KeywordTok}[1]{\textcolor[rgb]{0.13,0.29,0.53}{\textbf{#1}}}
\newcommand{\NormalTok}[1]{#1}
\newcommand{\OperatorTok}[1]{\textcolor[rgb]{0.81,0.36,0.00}{\textbf{#1}}}
\newcommand{\OtherTok}[1]{\textcolor[rgb]{0.56,0.35,0.01}{#1}}
\newcommand{\PreprocessorTok}[1]{\textcolor[rgb]{0.56,0.35,0.01}{\textit{#1}}}
\newcommand{\RegionMarkerTok}[1]{#1}
\newcommand{\SpecialCharTok}[1]{\textcolor[rgb]{0.00,0.00,0.00}{#1}}
\newcommand{\SpecialStringTok}[1]{\textcolor[rgb]{0.31,0.60,0.02}{#1}}
\newcommand{\StringTok}[1]{\textcolor[rgb]{0.31,0.60,0.02}{#1}}
\newcommand{\VariableTok}[1]{\textcolor[rgb]{0.00,0.00,0.00}{#1}}
\newcommand{\VerbatimStringTok}[1]{\textcolor[rgb]{0.31,0.60,0.02}{#1}}
\newcommand{\WarningTok}[1]{\textcolor[rgb]{0.56,0.35,0.01}{\textbf{\textit{#1}}}}
\usepackage{longtable,booktabs}
% Fix footnotes in tables (requires footnote package)
\IfFileExists{footnote.sty}{\usepackage{footnote}\makesavenoteenv{longtable}}{}
\usepackage{graphicx,grffile}
\makeatletter
\def\maxwidth{\ifdim\Gin@nat@width>\linewidth\linewidth\else\Gin@nat@width\fi}
\def\maxheight{\ifdim\Gin@nat@height>\textheight\textheight\else\Gin@nat@height\fi}
\makeatother
% Scale images if necessary, so that they will not overflow the page
% margins by default, and it is still possible to overwrite the defaults
% using explicit options in \includegraphics[width, height, ...]{}
\setkeys{Gin}{width=\maxwidth,height=\maxheight,keepaspectratio}
\setlength{\emergencystretch}{3em}  % prevent overfull lines
\providecommand{\tightlist}{%
  \setlength{\itemsep}{0pt}\setlength{\parskip}{0pt}}
\setcounter{secnumdepth}{0}
% Redefines (sub)paragraphs to behave more like sections
\ifx\paragraph\undefined\else
\let\oldparagraph\paragraph
\renewcommand{\paragraph}[1]{\oldparagraph{#1}\mbox{}}
\fi
\ifx\subparagraph\undefined\else
\let\oldsubparagraph\subparagraph
\renewcommand{\subparagraph}[1]{\oldsubparagraph{#1}\mbox{}}
\fi

% set default figure placement to htbp
\makeatletter
\def\fps@figure{htbp}
\makeatother


\title{CLV Exploratory}
\author{Josh Kong}
\date{10/13/2020}

\begin{document}
\maketitle

\begin{Shaded}
\begin{Highlighting}[]
\KeywordTok{library}\NormalTok{(tidyverse)}
\end{Highlighting}
\end{Shaded}

\begin{verbatim}
## -- Attaching packages ----------------------------------------------------------------------------- tidyverse 1.3.0 --
\end{verbatim}

\begin{verbatim}
## v ggplot2 3.3.2     v purrr   0.3.4
## v tibble  3.0.3     v dplyr   1.0.2
## v tidyr   1.1.2     v stringr 1.4.0
## v readr   1.3.1     v forcats 0.5.0
\end{verbatim}

\begin{verbatim}
## -- Conflicts -------------------------------------------------------------------------------- tidyverse_conflicts() --
## x dplyr::filter() masks stats::filter()
## x dplyr::lag()    masks stats::lag()
\end{verbatim}

\begin{Shaded}
\begin{Highlighting}[]
\KeywordTok{library}\NormalTok{(lubridate)}
\end{Highlighting}
\end{Shaded}

\begin{verbatim}
## 
## Attaching package: 'lubridate'
\end{verbatim}

\begin{verbatim}
## The following objects are masked from 'package:base':
## 
##     date, intersect, setdiff, union
\end{verbatim}

\begin{Shaded}
\begin{Highlighting}[]
\NormalTok{retail <-}\StringTok{ }\KeywordTok{read_csv}\NormalTok{(}\StringTok{"retail_cleaned.csv"}\NormalTok{)}
\end{Highlighting}
\end{Shaded}

\begin{verbatim}
## Warning: Missing column names filled in: 'X1' [1]
\end{verbatim}

\begin{verbatim}
## Parsed with column specification:
## cols(
##   X1 = col_double(),
##   InvoiceNo = col_character(),
##   StockCode = col_character(),
##   Quantity = col_double(),
##   InvoiceDate = col_date(format = ""),
##   UnitPrice = col_double(),
##   CustomerID = col_double(),
##   Country = col_character(),
##   Description = col_character(),
##   sales = col_double()
## )
\end{verbatim}

\begin{Shaded}
\begin{Highlighting}[]
\KeywordTok{theme_light}\NormalTok{()}
\end{Highlighting}
\end{Shaded}

\begin{verbatim}
## List of 93
##  $ line                      :List of 6
##   ..$ colour       : chr "black"
##   ..$ size         : num 0.5
##   ..$ linetype     : num 1
##   ..$ lineend      : chr "butt"
##   ..$ arrow        : logi FALSE
##   ..$ inherit.blank: logi TRUE
##   ..- attr(*, "class")= chr [1:2] "element_line" "element"
##  $ rect                      :List of 5
##   ..$ fill         : chr "white"
##   ..$ colour       : chr "black"
##   ..$ size         : num 0.5
##   ..$ linetype     : num 1
##   ..$ inherit.blank: logi TRUE
##   ..- attr(*, "class")= chr [1:2] "element_rect" "element"
##  $ text                      :List of 11
##   ..$ family       : chr ""
##   ..$ face         : chr "plain"
##   ..$ colour       : chr "black"
##   ..$ size         : num 11
##   ..$ hjust        : num 0.5
##   ..$ vjust        : num 0.5
##   ..$ angle        : num 0
##   ..$ lineheight   : num 0.9
##   ..$ margin       : 'margin' num [1:4] 0pt 0pt 0pt 0pt
##   .. ..- attr(*, "valid.unit")= int 8
##   .. ..- attr(*, "unit")= chr "pt"
##   ..$ debug        : logi FALSE
##   ..$ inherit.blank: logi TRUE
##   ..- attr(*, "class")= chr [1:2] "element_text" "element"
##  $ title                     : NULL
##  $ aspect.ratio              : NULL
##  $ axis.title                : NULL
##  $ axis.title.x              :List of 11
##   ..$ family       : NULL
##   ..$ face         : NULL
##   ..$ colour       : NULL
##   ..$ size         : NULL
##   ..$ hjust        : NULL
##   ..$ vjust        : num 1
##   ..$ angle        : NULL
##   ..$ lineheight   : NULL
##   ..$ margin       : 'margin' num [1:4] 2.75pt 0pt 0pt 0pt
##   .. ..- attr(*, "valid.unit")= int 8
##   .. ..- attr(*, "unit")= chr "pt"
##   ..$ debug        : NULL
##   ..$ inherit.blank: logi TRUE
##   ..- attr(*, "class")= chr [1:2] "element_text" "element"
##  $ axis.title.x.top          :List of 11
##   ..$ family       : NULL
##   ..$ face         : NULL
##   ..$ colour       : NULL
##   ..$ size         : NULL
##   ..$ hjust        : NULL
##   ..$ vjust        : num 0
##   ..$ angle        : NULL
##   ..$ lineheight   : NULL
##   ..$ margin       : 'margin' num [1:4] 0pt 0pt 2.75pt 0pt
##   .. ..- attr(*, "valid.unit")= int 8
##   .. ..- attr(*, "unit")= chr "pt"
##   ..$ debug        : NULL
##   ..$ inherit.blank: logi TRUE
##   ..- attr(*, "class")= chr [1:2] "element_text" "element"
##  $ axis.title.x.bottom       : NULL
##  $ axis.title.y              :List of 11
##   ..$ family       : NULL
##   ..$ face         : NULL
##   ..$ colour       : NULL
##   ..$ size         : NULL
##   ..$ hjust        : NULL
##   ..$ vjust        : num 1
##   ..$ angle        : num 90
##   ..$ lineheight   : NULL
##   ..$ margin       : 'margin' num [1:4] 0pt 2.75pt 0pt 0pt
##   .. ..- attr(*, "valid.unit")= int 8
##   .. ..- attr(*, "unit")= chr "pt"
##   ..$ debug        : NULL
##   ..$ inherit.blank: logi TRUE
##   ..- attr(*, "class")= chr [1:2] "element_text" "element"
##  $ axis.title.y.left         : NULL
##  $ axis.title.y.right        :List of 11
##   ..$ family       : NULL
##   ..$ face         : NULL
##   ..$ colour       : NULL
##   ..$ size         : NULL
##   ..$ hjust        : NULL
##   ..$ vjust        : num 0
##   ..$ angle        : num -90
##   ..$ lineheight   : NULL
##   ..$ margin       : 'margin' num [1:4] 0pt 0pt 0pt 2.75pt
##   .. ..- attr(*, "valid.unit")= int 8
##   .. ..- attr(*, "unit")= chr "pt"
##   ..$ debug        : NULL
##   ..$ inherit.blank: logi TRUE
##   ..- attr(*, "class")= chr [1:2] "element_text" "element"
##  $ axis.text                 :List of 11
##   ..$ family       : NULL
##   ..$ face         : NULL
##   ..$ colour       : chr "grey30"
##   ..$ size         : 'rel' num 0.8
##   ..$ hjust        : NULL
##   ..$ vjust        : NULL
##   ..$ angle        : NULL
##   ..$ lineheight   : NULL
##   ..$ margin       : NULL
##   ..$ debug        : NULL
##   ..$ inherit.blank: logi TRUE
##   ..- attr(*, "class")= chr [1:2] "element_text" "element"
##  $ axis.text.x               :List of 11
##   ..$ family       : NULL
##   ..$ face         : NULL
##   ..$ colour       : NULL
##   ..$ size         : NULL
##   ..$ hjust        : NULL
##   ..$ vjust        : num 1
##   ..$ angle        : NULL
##   ..$ lineheight   : NULL
##   ..$ margin       : 'margin' num [1:4] 2.2pt 0pt 0pt 0pt
##   .. ..- attr(*, "valid.unit")= int 8
##   .. ..- attr(*, "unit")= chr "pt"
##   ..$ debug        : NULL
##   ..$ inherit.blank: logi TRUE
##   ..- attr(*, "class")= chr [1:2] "element_text" "element"
##  $ axis.text.x.top           :List of 11
##   ..$ family       : NULL
##   ..$ face         : NULL
##   ..$ colour       : NULL
##   ..$ size         : NULL
##   ..$ hjust        : NULL
##   ..$ vjust        : num 0
##   ..$ angle        : NULL
##   ..$ lineheight   : NULL
##   ..$ margin       : 'margin' num [1:4] 0pt 0pt 2.2pt 0pt
##   .. ..- attr(*, "valid.unit")= int 8
##   .. ..- attr(*, "unit")= chr "pt"
##   ..$ debug        : NULL
##   ..$ inherit.blank: logi TRUE
##   ..- attr(*, "class")= chr [1:2] "element_text" "element"
##  $ axis.text.x.bottom        : NULL
##  $ axis.text.y               :List of 11
##   ..$ family       : NULL
##   ..$ face         : NULL
##   ..$ colour       : NULL
##   ..$ size         : NULL
##   ..$ hjust        : num 1
##   ..$ vjust        : NULL
##   ..$ angle        : NULL
##   ..$ lineheight   : NULL
##   ..$ margin       : 'margin' num [1:4] 0pt 2.2pt 0pt 0pt
##   .. ..- attr(*, "valid.unit")= int 8
##   .. ..- attr(*, "unit")= chr "pt"
##   ..$ debug        : NULL
##   ..$ inherit.blank: logi TRUE
##   ..- attr(*, "class")= chr [1:2] "element_text" "element"
##  $ axis.text.y.left          : NULL
##  $ axis.text.y.right         :List of 11
##   ..$ family       : NULL
##   ..$ face         : NULL
##   ..$ colour       : NULL
##   ..$ size         : NULL
##   ..$ hjust        : num 0
##   ..$ vjust        : NULL
##   ..$ angle        : NULL
##   ..$ lineheight   : NULL
##   ..$ margin       : 'margin' num [1:4] 0pt 0pt 0pt 2.2pt
##   .. ..- attr(*, "valid.unit")= int 8
##   .. ..- attr(*, "unit")= chr "pt"
##   ..$ debug        : NULL
##   ..$ inherit.blank: logi TRUE
##   ..- attr(*, "class")= chr [1:2] "element_text" "element"
##  $ axis.ticks                :List of 6
##   ..$ colour       : chr "grey70"
##   ..$ size         : 'rel' num 0.5
##   ..$ linetype     : NULL
##   ..$ lineend      : NULL
##   ..$ arrow        : logi FALSE
##   ..$ inherit.blank: logi TRUE
##   ..- attr(*, "class")= chr [1:2] "element_line" "element"
##  $ axis.ticks.x              : NULL
##  $ axis.ticks.x.top          : NULL
##  $ axis.ticks.x.bottom       : NULL
##  $ axis.ticks.y              : NULL
##  $ axis.ticks.y.left         : NULL
##  $ axis.ticks.y.right        : NULL
##  $ axis.ticks.length         : 'unit' num 2.75pt
##   ..- attr(*, "valid.unit")= int 8
##   ..- attr(*, "unit")= chr "pt"
##  $ axis.ticks.length.x       : NULL
##  $ axis.ticks.length.x.top   : NULL
##  $ axis.ticks.length.x.bottom: NULL
##  $ axis.ticks.length.y       : NULL
##  $ axis.ticks.length.y.left  : NULL
##  $ axis.ticks.length.y.right : NULL
##  $ axis.line                 : list()
##   ..- attr(*, "class")= chr [1:2] "element_blank" "element"
##  $ axis.line.x               : NULL
##  $ axis.line.x.top           : NULL
##  $ axis.line.x.bottom        : NULL
##  $ axis.line.y               : NULL
##  $ axis.line.y.left          : NULL
##  $ axis.line.y.right         : NULL
##  $ legend.background         :List of 5
##   ..$ fill         : NULL
##   ..$ colour       : logi NA
##   ..$ size         : NULL
##   ..$ linetype     : NULL
##   ..$ inherit.blank: logi TRUE
##   ..- attr(*, "class")= chr [1:2] "element_rect" "element"
##  $ legend.margin             : 'margin' num [1:4] 5.5pt 5.5pt 5.5pt 5.5pt
##   ..- attr(*, "valid.unit")= int 8
##   ..- attr(*, "unit")= chr "pt"
##  $ legend.spacing            : 'unit' num 11pt
##   ..- attr(*, "valid.unit")= int 8
##   ..- attr(*, "unit")= chr "pt"
##  $ legend.spacing.x          : NULL
##  $ legend.spacing.y          : NULL
##  $ legend.key                :List of 5
##   ..$ fill         : chr "white"
##   ..$ colour       : logi NA
##   ..$ size         : NULL
##   ..$ linetype     : NULL
##   ..$ inherit.blank: logi TRUE
##   ..- attr(*, "class")= chr [1:2] "element_rect" "element"
##  $ legend.key.size           : 'unit' num 1.2lines
##   ..- attr(*, "valid.unit")= int 3
##   ..- attr(*, "unit")= chr "lines"
##  $ legend.key.height         : NULL
##  $ legend.key.width          : NULL
##  $ legend.text               :List of 11
##   ..$ family       : NULL
##   ..$ face         : NULL
##   ..$ colour       : NULL
##   ..$ size         : 'rel' num 0.8
##   ..$ hjust        : NULL
##   ..$ vjust        : NULL
##   ..$ angle        : NULL
##   ..$ lineheight   : NULL
##   ..$ margin       : NULL
##   ..$ debug        : NULL
##   ..$ inherit.blank: logi TRUE
##   ..- attr(*, "class")= chr [1:2] "element_text" "element"
##  $ legend.text.align         : NULL
##  $ legend.title              :List of 11
##   ..$ family       : NULL
##   ..$ face         : NULL
##   ..$ colour       : NULL
##   ..$ size         : NULL
##   ..$ hjust        : num 0
##   ..$ vjust        : NULL
##   ..$ angle        : NULL
##   ..$ lineheight   : NULL
##   ..$ margin       : NULL
##   ..$ debug        : NULL
##   ..$ inherit.blank: logi TRUE
##   ..- attr(*, "class")= chr [1:2] "element_text" "element"
##  $ legend.title.align        : NULL
##  $ legend.position           : chr "right"
##  $ legend.direction          : NULL
##  $ legend.justification      : chr "center"
##  $ legend.box                : NULL
##  $ legend.box.just           : NULL
##  $ legend.box.margin         : 'margin' num [1:4] 0cm 0cm 0cm 0cm
##   ..- attr(*, "valid.unit")= int 1
##   ..- attr(*, "unit")= chr "cm"
##  $ legend.box.background     : list()
##   ..- attr(*, "class")= chr [1:2] "element_blank" "element"
##  $ legend.box.spacing        : 'unit' num 11pt
##   ..- attr(*, "valid.unit")= int 8
##   ..- attr(*, "unit")= chr "pt"
##  $ panel.background          :List of 5
##   ..$ fill         : chr "white"
##   ..$ colour       : logi NA
##   ..$ size         : NULL
##   ..$ linetype     : NULL
##   ..$ inherit.blank: logi TRUE
##   ..- attr(*, "class")= chr [1:2] "element_rect" "element"
##  $ panel.border              :List of 5
##   ..$ fill         : logi NA
##   ..$ colour       : chr "grey70"
##   ..$ size         : 'rel' num 1
##   ..$ linetype     : NULL
##   ..$ inherit.blank: logi TRUE
##   ..- attr(*, "class")= chr [1:2] "element_rect" "element"
##  $ panel.spacing             : 'unit' num 5.5pt
##   ..- attr(*, "valid.unit")= int 8
##   ..- attr(*, "unit")= chr "pt"
##  $ panel.spacing.x           : NULL
##  $ panel.spacing.y           : NULL
##  $ panel.grid                :List of 6
##   ..$ colour       : chr "grey87"
##   ..$ size         : NULL
##   ..$ linetype     : NULL
##   ..$ lineend      : NULL
##   ..$ arrow        : logi FALSE
##   ..$ inherit.blank: logi TRUE
##   ..- attr(*, "class")= chr [1:2] "element_line" "element"
##  $ panel.grid.major          :List of 6
##   ..$ colour       : NULL
##   ..$ size         : 'rel' num 0.5
##   ..$ linetype     : NULL
##   ..$ lineend      : NULL
##   ..$ arrow        : logi FALSE
##   ..$ inherit.blank: logi TRUE
##   ..- attr(*, "class")= chr [1:2] "element_line" "element"
##  $ panel.grid.minor          :List of 6
##   ..$ colour       : NULL
##   ..$ size         : 'rel' num 0.25
##   ..$ linetype     : NULL
##   ..$ lineend      : NULL
##   ..$ arrow        : logi FALSE
##   ..$ inherit.blank: logi TRUE
##   ..- attr(*, "class")= chr [1:2] "element_line" "element"
##  $ panel.grid.major.x        : NULL
##  $ panel.grid.major.y        : NULL
##  $ panel.grid.minor.x        : NULL
##  $ panel.grid.minor.y        : NULL
##  $ panel.ontop               : logi FALSE
##  $ plot.background           :List of 5
##   ..$ fill         : NULL
##   ..$ colour       : chr "white"
##   ..$ size         : NULL
##   ..$ linetype     : NULL
##   ..$ inherit.blank: logi TRUE
##   ..- attr(*, "class")= chr [1:2] "element_rect" "element"
##  $ plot.title                :List of 11
##   ..$ family       : NULL
##   ..$ face         : NULL
##   ..$ colour       : NULL
##   ..$ size         : 'rel' num 1.2
##   ..$ hjust        : num 0
##   ..$ vjust        : num 1
##   ..$ angle        : NULL
##   ..$ lineheight   : NULL
##   ..$ margin       : 'margin' num [1:4] 0pt 0pt 5.5pt 0pt
##   .. ..- attr(*, "valid.unit")= int 8
##   .. ..- attr(*, "unit")= chr "pt"
##   ..$ debug        : NULL
##   ..$ inherit.blank: logi TRUE
##   ..- attr(*, "class")= chr [1:2] "element_text" "element"
##  $ plot.title.position       : chr "panel"
##  $ plot.subtitle             :List of 11
##   ..$ family       : NULL
##   ..$ face         : NULL
##   ..$ colour       : NULL
##   ..$ size         : NULL
##   ..$ hjust        : num 0
##   ..$ vjust        : num 1
##   ..$ angle        : NULL
##   ..$ lineheight   : NULL
##   ..$ margin       : 'margin' num [1:4] 0pt 0pt 5.5pt 0pt
##   .. ..- attr(*, "valid.unit")= int 8
##   .. ..- attr(*, "unit")= chr "pt"
##   ..$ debug        : NULL
##   ..$ inherit.blank: logi TRUE
##   ..- attr(*, "class")= chr [1:2] "element_text" "element"
##  $ plot.caption              :List of 11
##   ..$ family       : NULL
##   ..$ face         : NULL
##   ..$ colour       : NULL
##   ..$ size         : 'rel' num 0.8
##   ..$ hjust        : num 1
##   ..$ vjust        : num 1
##   ..$ angle        : NULL
##   ..$ lineheight   : NULL
##   ..$ margin       : 'margin' num [1:4] 5.5pt 0pt 0pt 0pt
##   .. ..- attr(*, "valid.unit")= int 8
##   .. ..- attr(*, "unit")= chr "pt"
##   ..$ debug        : NULL
##   ..$ inherit.blank: logi TRUE
##   ..- attr(*, "class")= chr [1:2] "element_text" "element"
##  $ plot.caption.position     : chr "panel"
##  $ plot.tag                  :List of 11
##   ..$ family       : NULL
##   ..$ face         : NULL
##   ..$ colour       : NULL
##   ..$ size         : 'rel' num 1.2
##   ..$ hjust        : num 0.5
##   ..$ vjust        : num 0.5
##   ..$ angle        : NULL
##   ..$ lineheight   : NULL
##   ..$ margin       : NULL
##   ..$ debug        : NULL
##   ..$ inherit.blank: logi TRUE
##   ..- attr(*, "class")= chr [1:2] "element_text" "element"
##  $ plot.tag.position         : chr "topleft"
##  $ plot.margin               : 'margin' num [1:4] 5.5pt 5.5pt 5.5pt 5.5pt
##   ..- attr(*, "valid.unit")= int 8
##   ..- attr(*, "unit")= chr "pt"
##  $ strip.background          :List of 5
##   ..$ fill         : chr "grey70"
##   ..$ colour       : logi NA
##   ..$ size         : NULL
##   ..$ linetype     : NULL
##   ..$ inherit.blank: logi TRUE
##   ..- attr(*, "class")= chr [1:2] "element_rect" "element"
##  $ strip.background.x        : NULL
##  $ strip.background.y        : NULL
##  $ strip.placement           : chr "inside"
##  $ strip.text                :List of 11
##   ..$ family       : NULL
##   ..$ face         : NULL
##   ..$ colour       : chr "white"
##   ..$ size         : 'rel' num 0.8
##   ..$ hjust        : NULL
##   ..$ vjust        : NULL
##   ..$ angle        : NULL
##   ..$ lineheight   : NULL
##   ..$ margin       : 'margin' num [1:4] 4.4pt 4.4pt 4.4pt 4.4pt
##   .. ..- attr(*, "valid.unit")= int 8
##   .. ..- attr(*, "unit")= chr "pt"
##   ..$ debug        : NULL
##   ..$ inherit.blank: logi TRUE
##   ..- attr(*, "class")= chr [1:2] "element_text" "element"
##  $ strip.text.x              : NULL
##  $ strip.text.y              :List of 11
##   ..$ family       : NULL
##   ..$ face         : NULL
##   ..$ colour       : NULL
##   ..$ size         : NULL
##   ..$ hjust        : NULL
##   ..$ vjust        : NULL
##   ..$ angle        : num -90
##   ..$ lineheight   : NULL
##   ..$ margin       : NULL
##   ..$ debug        : NULL
##   ..$ inherit.blank: logi TRUE
##   ..- attr(*, "class")= chr [1:2] "element_text" "element"
##  $ strip.switch.pad.grid     : 'unit' num 2.75pt
##   ..- attr(*, "valid.unit")= int 8
##   ..- attr(*, "unit")= chr "pt"
##  $ strip.switch.pad.wrap     : 'unit' num 2.75pt
##   ..- attr(*, "valid.unit")= int 8
##   ..- attr(*, "unit")= chr "pt"
##  $ strip.text.y.left         :List of 11
##   ..$ family       : NULL
##   ..$ face         : NULL
##   ..$ colour       : NULL
##   ..$ size         : NULL
##   ..$ hjust        : NULL
##   ..$ vjust        : NULL
##   ..$ angle        : num 90
##   ..$ lineheight   : NULL
##   ..$ margin       : NULL
##   ..$ debug        : NULL
##   ..$ inherit.blank: logi TRUE
##   ..- attr(*, "class")= chr [1:2] "element_text" "element"
##  - attr(*, "class")= chr [1:2] "theme" "gg"
##  - attr(*, "complete")= logi TRUE
##  - attr(*, "validate")= logi TRUE
\end{verbatim}

\hypertarget{using-skimr-to-get-a-gist-for-how-the-dataset-looks}{%
\subsection{Using skimr to get a gist for how the dataset
looks}\label{using-skimr-to-get-a-gist-for-how-the-dataset-looks}}

\begin{Shaded}
\begin{Highlighting}[]
\NormalTok{skimr}\OperatorTok{::}\KeywordTok{skim}\NormalTok{(retail)}
\end{Highlighting}
\end{Shaded}

\begin{longtable}[]{@{}ll@{}}
\caption{Data summary}\tabularnewline
\toprule
\endhead
Name & retail\tabularnewline
Number of rows & 649107\tabularnewline
Number of columns & 10\tabularnewline
\_\_\_\_\_\_\_\_\_\_\_\_\_\_\_\_\_\_\_\_\_\_\_ &\tabularnewline
Column type frequency: &\tabularnewline
character & 4\tabularnewline
Date & 1\tabularnewline
numeric & 5\tabularnewline
\_\_\_\_\_\_\_\_\_\_\_\_\_\_\_\_\_\_\_\_\_\_\_\_ &\tabularnewline
Group variables & None\tabularnewline
\bottomrule
\end{longtable}

\textbf{Variable type: character}

\begin{longtable}[]{@{}lrrrrrrr@{}}
\toprule
skim\_variable & n\_missing & complete\_rate & min & max & empty &
n\_unique & whitespace\tabularnewline
\midrule
\endhead
InvoiceNo & 0 & 1 & 6 & 7 & 0 & 25787 & 0\tabularnewline
StockCode & 0 & 1 & 1 & 12 & 0 & 3957 & 0\tabularnewline
Country & 0 & 1 & 3 & 20 & 0 & 38 & 0\tabularnewline
Description & 0 & 1 & 2 & 35 & 0 & 4172 & 0\tabularnewline
\bottomrule
\end{longtable}

\textbf{Variable type: Date}

\begin{longtable}[]{@{}lrrlllr@{}}
\toprule
skim\_variable & n\_missing & complete\_rate & min & max & median &
n\_unique\tabularnewline
\midrule
\endhead
InvoiceDate & 0 & 1 & 2010-12-01 & 2011-12-09 & 2011-07-22 &
305\tabularnewline
\bottomrule
\end{longtable}

\textbf{Variable type: numeric}

\begin{longtable}[]{@{}lrrrrrrrrrl@{}}
\toprule
\begin{minipage}[b]{0.09\columnwidth}\raggedright
skim\_variable\strut
\end{minipage} & \begin{minipage}[b]{0.06\columnwidth}\raggedleft
n\_missing\strut
\end{minipage} & \begin{minipage}[b]{0.09\columnwidth}\raggedleft
complete\_rate\strut
\end{minipage} & \begin{minipage}[b]{0.06\columnwidth}\raggedleft
mean\strut
\end{minipage} & \begin{minipage}[b]{0.06\columnwidth}\raggedleft
sd\strut
\end{minipage} & \begin{minipage}[b]{0.07\columnwidth}\raggedleft
p0\strut
\end{minipage} & \begin{minipage}[b]{0.06\columnwidth}\raggedleft
p25\strut
\end{minipage} & \begin{minipage}[b]{0.06\columnwidth}\raggedleft
p50\strut
\end{minipage} & \begin{minipage}[b]{0.06\columnwidth}\raggedleft
p75\strut
\end{minipage} & \begin{minipage}[b]{0.06\columnwidth}\raggedleft
p100\strut
\end{minipage} & \begin{minipage}[b]{0.04\columnwidth}\raggedright
hist\strut
\end{minipage}\tabularnewline
\midrule
\endhead
\begin{minipage}[t]{0.09\columnwidth}\raggedright
X1\strut
\end{minipage} & \begin{minipage}[t]{0.06\columnwidth}\raggedleft
0\strut
\end{minipage} & \begin{minipage}[t]{0.09\columnwidth}\raggedleft
1.00\strut
\end{minipage} & \begin{minipage}[t]{0.06\columnwidth}\raggedleft
324554.00\strut
\end{minipage} & \begin{minipage}[t]{0.06\columnwidth}\raggedleft
187381.19\strut
\end{minipage} & \begin{minipage}[t]{0.07\columnwidth}\raggedleft
1.00\strut
\end{minipage} & \begin{minipage}[t]{0.06\columnwidth}\raggedleft
162277.50\strut
\end{minipage} & \begin{minipage}[t]{0.06\columnwidth}\raggedleft
324554.00\strut
\end{minipage} & \begin{minipage}[t]{0.06\columnwidth}\raggedleft
486830.50\strut
\end{minipage} & \begin{minipage}[t]{0.06\columnwidth}\raggedleft
649107.0\strut
\end{minipage} & \begin{minipage}[t]{0.04\columnwidth}\raggedright
▇▇▇▇▇\strut
\end{minipage}\tabularnewline
\begin{minipage}[t]{0.09\columnwidth}\raggedright
Quantity\strut
\end{minipage} & \begin{minipage}[t]{0.06\columnwidth}\raggedleft
0\strut
\end{minipage} & \begin{minipage}[t]{0.09\columnwidth}\raggedleft
1.00\strut
\end{minipage} & \begin{minipage}[t]{0.06\columnwidth}\raggedleft
9.84\strut
\end{minipage} & \begin{minipage}[t]{0.06\columnwidth}\raggedleft
202.48\strut
\end{minipage} & \begin{minipage}[t]{0.07\columnwidth}\raggedleft
-80995.00\strut
\end{minipage} & \begin{minipage}[t]{0.06\columnwidth}\raggedleft
1.00\strut
\end{minipage} & \begin{minipage}[t]{0.06\columnwidth}\raggedleft
4.00\strut
\end{minipage} & \begin{minipage}[t]{0.06\columnwidth}\raggedleft
10.00\strut
\end{minipage} & \begin{minipage}[t]{0.06\columnwidth}\raggedleft
80995.0\strut
\end{minipage} & \begin{minipage}[t]{0.04\columnwidth}\raggedright
▁▁▇▁▁\strut
\end{minipage}\tabularnewline
\begin{minipage}[t]{0.09\columnwidth}\raggedright
UnitPrice\strut
\end{minipage} & \begin{minipage}[t]{0.06\columnwidth}\raggedleft
0\strut
\end{minipage} & \begin{minipage}[t]{0.09\columnwidth}\raggedleft
1.00\strut
\end{minipage} & \begin{minipage}[t]{0.06\columnwidth}\raggedleft
4.40\strut
\end{minipage} & \begin{minipage}[t]{0.06\columnwidth}\raggedleft
88.44\strut
\end{minipage} & \begin{minipage}[t]{0.07\columnwidth}\raggedleft
-11062.06\strut
\end{minipage} & \begin{minipage}[t]{0.06\columnwidth}\raggedleft
1.25\strut
\end{minipage} & \begin{minipage}[t]{0.06\columnwidth}\raggedleft
2.10\strut
\end{minipage} & \begin{minipage}[t]{0.06\columnwidth}\raggedleft
4.13\strut
\end{minipage} & \begin{minipage}[t]{0.06\columnwidth}\raggedleft
38970.0\strut
\end{minipage} & \begin{minipage}[t]{0.04\columnwidth}\raggedright
▁▇▁▁▁\strut
\end{minipage}\tabularnewline
\begin{minipage}[t]{0.09\columnwidth}\raggedright
CustomerID\strut
\end{minipage} & \begin{minipage}[t]{0.06\columnwidth}\raggedleft
156547\strut
\end{minipage} & \begin{minipage}[t]{0.09\columnwidth}\raggedleft
0.76\strut
\end{minipage} & \begin{minipage}[t]{0.06\columnwidth}\raggedleft
15288.18\strut
\end{minipage} & \begin{minipage}[t]{0.06\columnwidth}\raggedleft
1711.58\strut
\end{minipage} & \begin{minipage}[t]{0.07\columnwidth}\raggedleft
12346.00\strut
\end{minipage} & \begin{minipage}[t]{0.06\columnwidth}\raggedleft
13952.00\strut
\end{minipage} & \begin{minipage}[t]{0.06\columnwidth}\raggedleft
15152.00\strut
\end{minipage} & \begin{minipage}[t]{0.06\columnwidth}\raggedleft
16791.00\strut
\end{minipage} & \begin{minipage}[t]{0.06\columnwidth}\raggedleft
18287.0\strut
\end{minipage} & \begin{minipage}[t]{0.04\columnwidth}\raggedright
▇▇▇▇▇\strut
\end{minipage}\tabularnewline
\begin{minipage}[t]{0.09\columnwidth}\raggedright
sales\strut
\end{minipage} & \begin{minipage}[t]{0.06\columnwidth}\raggedleft
0\strut
\end{minipage} & \begin{minipage}[t]{0.09\columnwidth}\raggedleft
1.00\strut
\end{minipage} & \begin{minipage}[t]{0.06\columnwidth}\raggedleft
19.21\strut
\end{minipage} & \begin{minipage}[t]{0.06\columnwidth}\raggedleft
352.03\strut
\end{minipage} & \begin{minipage}[t]{0.07\columnwidth}\raggedleft
-168469.60\strut
\end{minipage} & \begin{minipage}[t]{0.06\columnwidth}\raggedleft
3.75\strut
\end{minipage} & \begin{minipage}[t]{0.06\columnwidth}\raggedleft
9.95\strut
\end{minipage} & \begin{minipage}[t]{0.06\columnwidth}\raggedleft
17.70\strut
\end{minipage} & \begin{minipage}[t]{0.06\columnwidth}\raggedleft
168469.6\strut
\end{minipage} & \begin{minipage}[t]{0.04\columnwidth}\raggedright
▁▁▇▁▁\strut
\end{minipage}\tabularnewline
\bottomrule
\end{longtable}

The only data that we appear to be missing is CustomerID. This is
probably due to the customer only making a one time purchase without
making an account. I plan to filter these customers out later when I
predict the customer's future CLV values.

\begin{Shaded}
\begin{Highlighting}[]
\KeywordTok{range}\NormalTok{(retail}\OperatorTok{$}\NormalTok{InvoiceDate)}
\end{Highlighting}
\end{Shaded}

\begin{verbatim}
## [1] "2010-12-01" "2011-12-09"
\end{verbatim}

\begin{Shaded}
\begin{Highlighting}[]
\CommentTok{#the data only accounts for a month and half of data.}
\end{Highlighting}
\end{Shaded}

\begin{Shaded}
\begin{Highlighting}[]
\CommentTok{#looking at revenue by the dates}
\NormalTok{revenue_by_month <-}\StringTok{ }\NormalTok{retail }\OperatorTok\StringTok{ }
\StringTok{  }\KeywordTok{group_by}\NormalTok{(}\DataTypeTok{month =} \KeywordTok{floor_date}\NormalTok{(InvoiceDate, }\StringTok{"month"}\NormalTok{)) }\OperatorTok\StringTok{ }
\StringTok{  }\KeywordTok{summarise}\NormalTok{(}\DataTypeTok{revenue =} \KeywordTok{sum}\NormalTok{(sales)) }\OperatorTok\StringTok{ }
\StringTok{  }\KeywordTok{ungroup}\NormalTok{()}
\end{Highlighting}
\end{Shaded}

\begin{verbatim}
## `summarise()` ungrouping output (override with `.groups` argument)
\end{verbatim}

\begin{Shaded}
\begin{Highlighting}[]
\NormalTok{ revenue_by_month }\OperatorTok\StringTok{ }
\StringTok{  }\KeywordTok{ggplot}\NormalTok{(}\KeywordTok{aes}\NormalTok{(month, revenue)) }\OperatorTok{+}\StringTok{ }
\StringTok{  }\KeywordTok{geom_point}\NormalTok{()}\OperatorTok{+}
\StringTok{  }\KeywordTok{geom_line}\NormalTok{() }\OperatorTok{+}
\StringTok{   }\KeywordTok{labs}\NormalTok{(}\DataTypeTok{x =} \StringTok{""}\NormalTok{, }\DataTypeTok{y =} \StringTok{"Revenue"}\NormalTok{, }\DataTypeTok{title =} \StringTok{"Revenue by Month"}\NormalTok{)}
\end{Highlighting}
\end{Shaded}

\includegraphics{clv_exploring_files/figure-latex/unnamed-chunk-4-1.pdf}

\begin{Shaded}
\begin{Highlighting}[]
\CommentTok{#there is a very large spike in revenue during November. This is probably due to Black Friday sales, or early Christmas shopping.}
\CommentTok{#december 2011 has low numbers in sales. this is most likely due to the data not being complete}
\end{Highlighting}
\end{Shaded}

\begin{Shaded}
\begin{Highlighting}[]
\CommentTok{#looking at the monthly growth rate}
\NormalTok{monthly_growth <-}\StringTok{ }\KeywordTok{rep}\NormalTok{(}\OtherTok{NA}\NormalTok{, }\KeywordTok{nrow}\NormalTok{(revenue_by_month))}
\NormalTok{monthly_growth[}\DecValTok{1}\NormalTok{] <-}\StringTok{ }\DecValTok{0}
\ControlFlowTok{for}\NormalTok{(i }\ControlFlowTok{in} \DecValTok{2}\OperatorTok{:}\KeywordTok{nrow}\NormalTok{(revenue_by_month}\DecValTok{-1}\NormalTok{))}
\NormalTok{\{}
\NormalTok{  monthly_growth[i] <-}\StringTok{ }\NormalTok{(revenue_by_month}\OperatorTok{$}\NormalTok{revenue[i] }\OperatorTok{-}\StringTok{ }\NormalTok{revenue_by_month}\OperatorTok{$}\NormalTok{revenue[i}\DecValTok{-1}\NormalTok{])}\OperatorTok{/}\NormalTok{revenue_by_month}\OperatorTok{$}\NormalTok{revenue[i}\DecValTok{-1}\NormalTok{]}
\NormalTok{\}}
\NormalTok{revenue_by_month}\OperatorTok{$}\NormalTok{monthly_growth <-}\StringTok{ }\NormalTok{monthly_growth}
\NormalTok{revenue_by_month }\OperatorTok\StringTok{ }
\StringTok{  }\KeywordTok{filter}\NormalTok{(month }\OperatorTok{!=}\StringTok{ "2011-12-01"}\NormalTok{) }\OperatorTok\StringTok{ }
\StringTok{  }\KeywordTok{ggplot}\NormalTok{(}\KeywordTok{aes}\NormalTok{(month, monthly_growth)) }\OperatorTok{+}\StringTok{ }
\StringTok{  }\KeywordTok{geom_point}\NormalTok{() }\OperatorTok{+}\StringTok{ }
\StringTok{  }\KeywordTok{geom_line}\NormalTok{()}\OperatorTok{+}
\StringTok{  }\KeywordTok{geom_hline}\NormalTok{(}\DataTypeTok{yintercept =} \DecValTok{0}\NormalTok{, }\DataTypeTok{lty =} \DecValTok{2}\NormalTok{) }\OperatorTok{+}\StringTok{ }
\StringTok{  }\KeywordTok{labs}\NormalTok{(}\DataTypeTok{title =} \StringTok{"Monthly Growth"}\NormalTok{, }\DataTypeTok{y =} \StringTok{""}\NormalTok{) }\OperatorTok{+}\StringTok{ }
\StringTok{  }\KeywordTok{scale_y_continuous}\NormalTok{(}\DataTypeTok{labels =}\NormalTok{ scales}\OperatorTok{::}\KeywordTok{percent_format}\NormalTok{())}
\end{Highlighting}
\end{Shaded}

\includegraphics{clv_exploring_files/figure-latex/unnamed-chunk-5-1.pdf}

\begin{Shaded}
\begin{Highlighting}[]
\CommentTok{#I'm excluding 2011 December from this graph because the data is still incomplete.}
\end{Highlighting}
\end{Shaded}

\begin{Shaded}
\begin{Highlighting}[]
\CommentTok{#looking at monthly active customers}
\NormalTok{retail }\OperatorTok\StringTok{ }
\StringTok{  }\KeywordTok{filter}\NormalTok{(}\OperatorTok{!}\KeywordTok{is.na}\NormalTok{(CustomerID)) }\OperatorTok\StringTok{ }
\StringTok{  }\KeywordTok{group_by}\NormalTok{(}\DataTypeTok{month =} \KeywordTok{floor_date}\NormalTok{(InvoiceDate, }\StringTok{"month"}\NormalTok{), CustomerID) }\OperatorTok\StringTok{ }
\StringTok{  }\KeywordTok{summarise}\NormalTok{(}\DataTypeTok{count =} \KeywordTok{n}\NormalTok{()) }\OperatorTok\StringTok{ }
\StringTok{  }\KeywordTok{ungroup}\NormalTok{() }\OperatorTok\StringTok{ }
\StringTok{  }\KeywordTok{count}\NormalTok{(month) }\OperatorTok\StringTok{ }
\StringTok{  }\KeywordTok{filter}\NormalTok{(month }\OperatorTok{!=}\StringTok{ "2011-12-01"}\NormalTok{) }\OperatorTok\StringTok{ }
\StringTok{  }\KeywordTok{ggplot}\NormalTok{(}\KeywordTok{aes}\NormalTok{(month, n)) }\OperatorTok{+}\StringTok{ }
\StringTok{  }\KeywordTok{geom_point}\NormalTok{() }\OperatorTok{+}\StringTok{ }
\StringTok{  }\KeywordTok{geom_line}\NormalTok{()}\OperatorTok{+}\StringTok{ }
\StringTok{  }\KeywordTok{labs}\NormalTok{(}\DataTypeTok{y =} \StringTok{"Number of Active Users"}\NormalTok{, }\DataTypeTok{x =} \StringTok{""}\NormalTok{, }\DataTypeTok{title =} \StringTok{"Number of Active Users per Month"}\NormalTok{)}
\end{Highlighting}
\end{Shaded}

\begin{verbatim}
## `summarise()` regrouping output by 'month' (override with `.groups` argument)
\end{verbatim}

\includegraphics{clv_exploring_files/figure-latex/unnamed-chunk-6-1.pdf}

\begin{Shaded}
\begin{Highlighting}[]
\CommentTok{#looks like there is consistent growth in the number of active users per month}
\end{Highlighting}
\end{Shaded}

\begin{Shaded}
\begin{Highlighting}[]
\CommentTok{#taking a look a new customers per month}
\NormalTok{retail }\OperatorTok\StringTok{ }
\StringTok{  }\KeywordTok{distinct}\NormalTok{(CustomerID, InvoiceDate) }\OperatorTok\StringTok{ }
\StringTok{  }\KeywordTok{group_by}\NormalTok{(CustomerID) }\OperatorTok\StringTok{ }
\StringTok{  }\KeywordTok{summarise}\NormalTok{(}\DataTypeTok{first_purchase_date =} \KeywordTok{min}\NormalTok{(InvoiceDate)) }\OperatorTok\StringTok{ }
\StringTok{  }\KeywordTok{ungroup}\NormalTok{() }\OperatorTok\StringTok{ }
\StringTok{  }\KeywordTok{count}\NormalTok{(first_purchase_date) }\OperatorTok\StringTok{ }
\StringTok{  }\KeywordTok{group_by}\NormalTok{(}\DataTypeTok{month =} \KeywordTok{floor_date}\NormalTok{(first_purchase_date, }\StringTok{"month"}\NormalTok{)) }\OperatorTok\StringTok{ }
\StringTok{  }\KeywordTok{summarise}\NormalTok{(}\DataTypeTok{new_customers =} \KeywordTok{sum}\NormalTok{(n)) }\OperatorTok\StringTok{ }
\StringTok{  }\KeywordTok{filter}\NormalTok{(month }\OperatorTok{!=}\StringTok{ "2011-12-01"}\NormalTok{, month }\OperatorTok{!=}\StringTok{ "2010-12-01"}\NormalTok{) }\OperatorTok\StringTok{  }\CommentTok{#filtering out december because data is still incomplete}
\StringTok{  }\KeywordTok{ggplot}\NormalTok{(}\KeywordTok{aes}\NormalTok{(month, new_customers)) }\OperatorTok{+}\StringTok{ }
\StringTok{  }\KeywordTok{geom_point}\NormalTok{() }\OperatorTok{+}\StringTok{ }
\StringTok{  }\KeywordTok{geom_line}\NormalTok{()}
\end{Highlighting}
\end{Shaded}

\begin{verbatim}
## `summarise()` ungrouping output (override with `.groups` argument)
## `summarise()` ungrouping output (override with `.groups` argument)
\end{verbatim}

\includegraphics{clv_exploring_files/figure-latex/unnamed-chunk-7-1.pdf}

\begin{Shaded}
\begin{Highlighting}[]
\CommentTok{#we didn't have many new customers from June through August. Starting September there was a spike in new customers once again}
\end{Highlighting}
\end{Shaded}

\hypertarget{rfm-recency-frequency-monetary-value-clustering}{%
\subsection{RFM (recency, frequency, monetary value)
Clustering}\label{rfm-recency-frequency-monetary-value-clustering}}

\begin{Shaded}
\begin{Highlighting}[]
\KeywordTok{library}\NormalTok{(cluster)}
\KeywordTok{library}\NormalTok{(factoextra)}
\end{Highlighting}
\end{Shaded}

\begin{verbatim}
## Welcome! Want to learn more? See two factoextra-related books at https://goo.gl/ve3WBa
\end{verbatim}

Calculating recency. Recency score is equal to the most recent purchase
subtracted by the customer's most recent purchase.

\begin{Shaded}
\begin{Highlighting}[]
\CommentTok{#going to calculate recency by subtracting the customer's most recent transcation by the max of the invoice date}
\NormalTok{maxdate <-}\StringTok{ }\KeywordTok{max}\NormalTok{(retail}\OperatorTok{$}\NormalTok{InvoiceDate)}

\NormalTok{recency_scores <-}\StringTok{ }\NormalTok{retail }\OperatorTok\StringTok{ }
\StringTok{  }\KeywordTok{filter}\NormalTok{(}\OperatorTok{!}\KeywordTok{is.na}\NormalTok{(CustomerID)) }\OperatorTok\StringTok{ }
\StringTok{  }\KeywordTok{group_by}\NormalTok{(CustomerID) }\OperatorTok\StringTok{ }
\StringTok{  }\KeywordTok{summarise}\NormalTok{(}\DataTypeTok{recency =}\NormalTok{ maxdate }\OperatorTok{-}\StringTok{ }\KeywordTok{max}\NormalTok{(InvoiceDate)) }
\end{Highlighting}
\end{Shaded}

\begin{verbatim}
## `summarise()` ungrouping output (override with `.groups` argument)
\end{verbatim}

\begin{Shaded}
\begin{Highlighting}[]
\CommentTok{#distribution of recency scores}
\NormalTok{recency_scores }\OperatorTok\StringTok{ }
\StringTok{  }\KeywordTok{ggplot}\NormalTok{(}\KeywordTok{aes}\NormalTok{(recency)) }\OperatorTok{+}\StringTok{ }
\StringTok{  }\KeywordTok{geom_histogram}\NormalTok{()}
\end{Highlighting}
\end{Shaded}

\begin{verbatim}
## Don't know how to automatically pick scale for object of type difftime. Defaulting to continuous.
\end{verbatim}

\begin{verbatim}
## `stat_bin()` using `bins = 30`. Pick better value with `binwidth`.
\end{verbatim}

\includegraphics{clv_exploring_files/figure-latex/unnamed-chunk-9-1.pdf}

Choosing the number of clusters for recency and then using kmeans
clustering

\begin{Shaded}
\begin{Highlighting}[]
\CommentTok{#elbow method to see optimal number of clusters}
\KeywordTok{fviz_nbclust}\NormalTok{(}\KeywordTok{as.matrix}\NormalTok{(recency_scores}\OperatorTok{$}\NormalTok{recency), kmeans, }\DataTypeTok{method =} \StringTok{"wss"}\NormalTok{)}
\end{Highlighting}
\end{Shaded}

\includegraphics{clv_exploring_files/figure-latex/unnamed-chunk-10-1.pdf}

\begin{Shaded}
\begin{Highlighting}[]
\CommentTok{#looking at the elbow plot, it seems it would be optimal to do 4 clusters}

\NormalTok{recency_cluster <-}\StringTok{ }\KeywordTok{kmeans}\NormalTok{(recency_scores}\OperatorTok{$}\NormalTok{recency, }\DecValTok{4}\NormalTok{)}
\NormalTok{recency_center <-}\StringTok{ }\KeywordTok{sort}\NormalTok{(}\KeywordTok{desc}\NormalTok{(recency_cluster}\OperatorTok{$}\NormalTok{centers)) }\OperatorTok{*}\StringTok{ }\DecValTok{-1}
\NormalTok{recency_cluster <-}\StringTok{ }\KeywordTok{kmeans}\NormalTok{(recency_scores}\OperatorTok{$}\NormalTok{recency, }\DataTypeTok{centers =}\NormalTok{ recency_center) }
\CommentTok{#doing it twice to order the clusters by the cluster means}
\NormalTok{recency_scores}\OperatorTok{$}\NormalTok{r_clus <-}\StringTok{ }\NormalTok{recency_cluster}\OperatorTok{$}\NormalTok{cluster}

\CommentTok{#the higher the score, the more recently they've purchased an item}
\end{Highlighting}
\end{Shaded}

Calculating Frequency. Total number of order by a certain customer.

\begin{Shaded}
\begin{Highlighting}[]
\CommentTok{#We'll be counting each invoice number as one order.}
\NormalTok{frequency_scores <-}\StringTok{ }\NormalTok{retail }\OperatorTok\StringTok{ }
\StringTok{  }\KeywordTok{filter}\NormalTok{(}\OperatorTok{!}\KeywordTok{is.na}\NormalTok{(CustomerID)) }\OperatorTok\StringTok{ }
\StringTok{  }\KeywordTok{count}\NormalTok{(CustomerID, InvoiceNo) }\OperatorTok\StringTok{ }
\StringTok{  }\KeywordTok{count}\NormalTok{(CustomerID, }\DataTypeTok{sort =} \OtherTok{TRUE}\NormalTok{)}

\CommentTok{#distribution of the frequency scores}
\NormalTok{retail }\OperatorTok\StringTok{ }
\StringTok{  }\KeywordTok{filter}\NormalTok{(}\OperatorTok{!}\KeywordTok{is.na}\NormalTok{(CustomerID)) }\OperatorTok\StringTok{ }
\StringTok{  }\KeywordTok{count}\NormalTok{(CustomerID, InvoiceNo) }\OperatorTok\StringTok{ }
\StringTok{  }\KeywordTok{count}\NormalTok{(CustomerID, }\DataTypeTok{sort =} \OtherTok{TRUE}\NormalTok{) }\OperatorTok\StringTok{ }
\StringTok{  }\KeywordTok{ggplot}\NormalTok{(}\KeywordTok{aes}\NormalTok{(n)) }\OperatorTok{+}\StringTok{ }
\StringTok{  }\KeywordTok{geom_histogram}\NormalTok{(}\DataTypeTok{binwidth =} \DecValTok{15}\NormalTok{)}
\end{Highlighting}
\end{Shaded}

\includegraphics{clv_exploring_files/figure-latex/unnamed-chunk-11-1.pdf}

Choosing the optimal number of clusters for frequency and doing k means
clustering.

\begin{Shaded}
\begin{Highlighting}[]
\KeywordTok{fviz_nbclust}\NormalTok{(}\KeywordTok{as.matrix}\NormalTok{(frequency_scores}\OperatorTok{$}\NormalTok{n), kmeans, }\DataTypeTok{method =} \StringTok{"wss"}\NormalTok{)}
\end{Highlighting}
\end{Shaded}

\includegraphics{clv_exploring_files/figure-latex/unnamed-chunk-12-1.pdf}

\begin{Shaded}
\begin{Highlighting}[]
\CommentTok{#looking at the elbow plot, it seems it would be optimal to do 5 clusters. However, to be consistent with recency, I will do 4 clusters once again.}

\NormalTok{frequency_cluster <-}\StringTok{ }\KeywordTok{kmeans}\NormalTok{(frequency_scores}\OperatorTok{$}\NormalTok{n, }\DecValTok{4}\NormalTok{)}
\NormalTok{frequency_center <-}\StringTok{ }\KeywordTok{sort}\NormalTok{(frequency_cluster}\OperatorTok{$}\NormalTok{centers)}
\NormalTok{frequency_cluster <-}\StringTok{ }\KeywordTok{kmeans}\NormalTok{(frequency_scores}\OperatorTok{$}\NormalTok{n, }\DataTypeTok{centers =}\NormalTok{ frequency_center) }
\CommentTok{#doing it twice to order the clusters by the cluster means}
\NormalTok{frequency_scores}\OperatorTok{$}\NormalTok{f_clus<-}\StringTok{ }\NormalTok{frequency_cluster}\OperatorTok{$}\NormalTok{cluster}
\end{Highlighting}
\end{Shaded}

Calculating monetary value

\begin{Shaded}
\begin{Highlighting}[]
\CommentTok{#we are going to be calculating the total revenue each customer brought in}
\NormalTok{money_scores <-}\StringTok{ }\NormalTok{retail }\OperatorTok\StringTok{ }
\StringTok{  }\KeywordTok{filter}\NormalTok{(}\OperatorTok{!}\KeywordTok{is.na}\NormalTok{(CustomerID)) }\OperatorTok\StringTok{ }
\StringTok{  }\KeywordTok{group_by}\NormalTok{(CustomerID) }\OperatorTok\StringTok{ }
\StringTok{  }\KeywordTok{summarise}\NormalTok{(}\DataTypeTok{total_revenue =} \KeywordTok{sum}\NormalTok{(sales))}
\end{Highlighting}
\end{Shaded}

\begin{verbatim}
## `summarise()` ungrouping output (override with `.groups` argument)
\end{verbatim}

\begin{Shaded}
\begin{Highlighting}[]
\CommentTok{#looking at the distribution of monetary value}
\NormalTok{money_scores }\OperatorTok\StringTok{ }
\StringTok{  }\KeywordTok{ggplot}\NormalTok{(}\KeywordTok{aes}\NormalTok{(total_revenue)) }\OperatorTok{+}\StringTok{ }
\StringTok{  }\KeywordTok{geom_histogram}\NormalTok{()}
\end{Highlighting}
\end{Shaded}

\begin{verbatim}
## `stat_bin()` using `bins = 30`. Pick better value with `binwidth`.
\end{verbatim}

\includegraphics{clv_exploring_files/figure-latex/unnamed-chunk-13-1.pdf}

Choosing optimal number of clusters for monetary value and doing k means
clustering

\begin{Shaded}
\begin{Highlighting}[]
\KeywordTok{fviz_nbclust}\NormalTok{(}\KeywordTok{as.matrix}\NormalTok{(money_scores}\OperatorTok{$}\NormalTok{total_revenue), kmeans, }\DataTypeTok{method =} \StringTok{"wss"}\NormalTok{)}
\end{Highlighting}
\end{Shaded}

\includegraphics{clv_exploring_files/figure-latex/unnamed-chunk-14-1.pdf}

\begin{Shaded}
\begin{Highlighting}[]
\CommentTok{#looking at the elbow plot, it seems it would be optimal to do 4 clusters. }

\NormalTok{money_cluster <-}\StringTok{ }\KeywordTok{kmeans}\NormalTok{(money_scores}\OperatorTok{$}\NormalTok{total_revenue, }\DecValTok{4}\NormalTok{)}
\NormalTok{money_center <-}\StringTok{ }\KeywordTok{sort}\NormalTok{(money_cluster}\OperatorTok{$}\NormalTok{centers)}
\NormalTok{money_cluster <-}\StringTok{ }\KeywordTok{kmeans}\NormalTok{(money_scores}\OperatorTok{$}\NormalTok{total_revenue, }\DataTypeTok{centers =}\NormalTok{ money_center) }
\CommentTok{#doing it twice to order the clusters by the cluster means}
\NormalTok{money_scores}\OperatorTok{$}\NormalTok{m_clus <-}\StringTok{ }\NormalTok{money_cluster}\OperatorTok{$}\NormalTok{cluster}
\end{Highlighting}
\end{Shaded}

Combining all the scores together

\begin{Shaded}
\begin{Highlighting}[]
\NormalTok{frequency_scores <-}\StringTok{ }\NormalTok{frequency_scores }\OperatorTok\StringTok{ }
\StringTok{  }\KeywordTok{arrange}\NormalTok{(CustomerID)}

\NormalTok{rfm_scores <-}\StringTok{ }\KeywordTok{as_tibble}\NormalTok{(}\KeywordTok{data.frame}\NormalTok{(recency_scores, frequency_scores[}\OperatorTok{-}\DecValTok{1}\NormalTok{], money_scores[}\OperatorTok{-}\DecValTok{1}\NormalTok{]))}
\KeywordTok{colnames}\NormalTok{(rfm_scores) <-}\StringTok{ }\KeywordTok{c}\NormalTok{(}\StringTok{"CustomerID"}\NormalTok{,}\StringTok{"recency"}\NormalTok{,}\StringTok{"r_clus"}\NormalTok{,}\StringTok{"frequency"}\NormalTok{,}\StringTok{"f_clus"}\NormalTok{,}\StringTok{"money_value"}\NormalTok{,}\StringTok{"m_clus"}\NormalTok{)}
\NormalTok{rfm_scores <-}\StringTok{ }\NormalTok{rfm_scores }\OperatorTok\StringTok{ }
\StringTok{  }\KeywordTok{group_by}\NormalTok{(CustomerID) }\OperatorTok\StringTok{ }
\StringTok{  }\KeywordTok{mutate}\NormalTok{(}\DataTypeTok{score =} \KeywordTok{sum}\NormalTok{(r_clus,f_clus,m_clus)) }\OperatorTok\StringTok{ }
\StringTok{  }\KeywordTok{ungroup}\NormalTok{()}


\CommentTok{#looking at the distribution of scores. Nobody got a perfect score of 12}
\NormalTok{rfm_scores }\OperatorTok\StringTok{ }
\StringTok{  }\KeywordTok{count}\NormalTok{(score) }\OperatorTok\StringTok{ }
\StringTok{  }\KeywordTok{ggplot}\NormalTok{(}\KeywordTok{aes}\NormalTok{(}\KeywordTok{as.factor}\NormalTok{(score), n)) }\OperatorTok{+}\StringTok{ }
\StringTok{  }\KeywordTok{geom_col}\NormalTok{() }\OperatorTok{+}
\StringTok{  }\KeywordTok{labs}\NormalTok{(}\DataTypeTok{title=} \StringTok{"Distribution of Scores"}\NormalTok{, }\DataTypeTok{x =} \StringTok{"Score"}\NormalTok{, }\DataTypeTok{y=} \StringTok{""}\NormalTok{)}
\end{Highlighting}
\end{Shaded}

\includegraphics{clv_exploring_files/figure-latex/unnamed-chunk-15-1.pdf}

The scores range from 3-12.

I will value 3-4 as low value customers. 5-7 as mid value customers.
8-11 as high value customers.

\begin{Shaded}
\begin{Highlighting}[]
\ControlFlowTok{for}\NormalTok{(i }\ControlFlowTok{in} \DecValTok{1}\OperatorTok{:}\KeywordTok{nrow}\NormalTok{(rfm_scores))}
\NormalTok{\{}
  \ControlFlowTok{if}\NormalTok{(rfm_scores}\OperatorTok{$}\NormalTok{score[i] }\OperatorTok{<=}\StringTok{ }\DecValTok{4}\NormalTok{)}
\NormalTok{  \{}
\NormalTok{    rfm_scores}\OperatorTok{$}\NormalTok{value[i] =}\StringTok{ "low"}
\NormalTok{  \}}
  \ControlFlowTok{else} \ControlFlowTok{if}\NormalTok{(rfm_scores}\OperatorTok{$}\NormalTok{score[i] }\OperatorTok{>}\StringTok{ }\DecValTok{4} \OperatorTok{&&}\StringTok{ }\NormalTok{rfm_scores}\OperatorTok{$}\NormalTok{score[i] }\OperatorTok{<}\DecValTok{8}\NormalTok{)}
\NormalTok{  \{}
\NormalTok{    rfm_scores}\OperatorTok{$}\NormalTok{value[i] =}\StringTok{ "mid"}
\NormalTok{  \}}
  \ControlFlowTok{else}
\NormalTok{  \{}
\NormalTok{    rfm_scores}\OperatorTok{$}\NormalTok{value[i] =}\StringTok{ "high"}
\NormalTok{  \}}
\NormalTok{\}}
\end{Highlighting}
\end{Shaded}

\begin{verbatim}
## Warning: Unknown or uninitialised column: `value`.
\end{verbatim}

\begin{Shaded}
\begin{Highlighting}[]
\NormalTok{rfm_scores}\OperatorTok{$}\NormalTok{value <-}\StringTok{ }\KeywordTok{as.factor}\NormalTok{(rfm_scores}\OperatorTok{$}\NormalTok{value)}
\end{Highlighting}
\end{Shaded}

Taking a look at how our valued customers are distributed in a
scatterplot.

\begin{Shaded}
\begin{Highlighting}[]
\NormalTok{rfm_scores }\OperatorTok\StringTok{ }
\StringTok{  }\KeywordTok{ggplot}\NormalTok{(}\KeywordTok{aes}\NormalTok{(recency, frequency, }\DataTypeTok{color =}\NormalTok{ value)) }\OperatorTok{+}
\StringTok{  }\KeywordTok{geom_point}\NormalTok{()}
\end{Highlighting}
\end{Shaded}

\begin{verbatim}
## Don't know how to automatically pick scale for object of type difftime. Defaulting to continuous.
\end{verbatim}

\includegraphics{clv_exploring_files/figure-latex/unnamed-chunk-17-1.pdf}

\begin{Shaded}
\begin{Highlighting}[]
\NormalTok{rfm_scores }\OperatorTok\StringTok{ }
\StringTok{  }\KeywordTok{ggplot}\NormalTok{(}\KeywordTok{aes}\NormalTok{(recency, money_value, }\DataTypeTok{color =}\NormalTok{ value))}\OperatorTok{+}
\StringTok{  }\KeywordTok{geom_point}\NormalTok{()}
\end{Highlighting}
\end{Shaded}

\begin{verbatim}
## Don't know how to automatically pick scale for object of type difftime. Defaulting to continuous.
\end{verbatim}

\includegraphics{clv_exploring_files/figure-latex/unnamed-chunk-17-2.pdf}

\begin{Shaded}
\begin{Highlighting}[]
\NormalTok{rfm_scores }\OperatorTok\StringTok{ }
\StringTok{  }\KeywordTok{ggplot}\NormalTok{(}\KeywordTok{aes}\NormalTok{(frequency,money_value, }\DataTypeTok{color =}\NormalTok{ value)) }\OperatorTok{+}\StringTok{ }
\StringTok{  }\KeywordTok{geom_point}\NormalTok{()}
\end{Highlighting}
\end{Shaded}

\includegraphics{clv_exploring_files/figure-latex/unnamed-chunk-17-3.pdf}

Saving the scores dataset

\begin{Shaded}
\begin{Highlighting}[]
\KeywordTok{write.csv}\NormalTok{(rfm_scores, }\StringTok{"rfm_scores.csv"}\NormalTok{)}
\end{Highlighting}
\end{Shaded}

\end{document}
